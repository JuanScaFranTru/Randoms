\section{Conclusión}

    \par Se presentó el problema de determinar el tiempo de falla esperado de un lavadero
    en función de la cantidad de operarios y lavarropas de repuesto. Se construyó un
    modelo y a partir del mismo se obtuvieron datos simulados correspondientes al
    tiempo de falla del sistema. Dado que resulta de gran interés saber cómo
    maximizar el tiempo que tarda el sistema en fallar en función de estos
    parámetros, los distintos experimentos realizados varían estos parámetros.


    \par Agregar una máquina de repuesto incrementa el tiempo medio de fallo del sistema
    en un 50\%. Agregar un operario incrementa el tiempo medio de fallo del sistema
    en un 107\%. Claramente, agregar una máquina aumenta más el tiempo medio de fallo
    del sistema que agregar un operario. Estos resultados no deben extrapolarse a
    otras situaciones en las cuáles los parámetros sean distintos (ver Resultados).

    \par Es importante destacar que el algoritmo propuesto puede ser útil para realizar
    otros experimentos modificando los parámetros del mismo.
