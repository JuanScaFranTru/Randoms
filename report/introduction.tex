\section{\textbf{Introducción}}

    \par Se tiene un lavadero con una cierta cantidad de máquinas que deben
    estar funcionando en todo momento para que éste sea operativo. Dado
    que las máquinas se descomponen cada una cierta cantidad de tiempo,
    resulta relevante poder predecir el tiempo que tarda el sistema en fallar
    y determinar si es mejor aumentar la cantidad de máquinas de repuesto o
    aumentar la cantidad de operarios que las reparan.

    \par Claramente, determinar analíticamente el tiempo medio que tarda el
    sistema en fallar es muy dificil. Por esto es que se realizaron simulaciones
    para determinar el tiempo medio que transcurre hasta que el lavadero deja
    de ser operativo y la desviación estándar.

    \par Por otra parte, estas simulaciones permiten determinar si es más
    conveniente aumentar la cantidad de máquinas de repuesto o la cantidad de
    operarios.


    \subsection{Modelo}

    \par El lavadero cuenta con $N$ máquinas lavadoras en servicio y $S$
    máquinas de respuesto, todas ellas de idéntica marca, modelo y antigüedad.
    Además el lavadero cuenta con los servicios de técnicos que reparan
    las máquinas cuando éstas se rompen. Los técnicos trabajan en paralelo y
    cada uno se encarga de arreglar máquinas de a una por vez.
    Todos los tiempos de funcionamiento de las máquinas hasta descomponerse
    son variables aleatorias independientes exponenciales con un tiempo medio de
    fallar de $T_f$.
    El tiempo de reparación de una máquina que ingresa al taller es una
    variable exponencial con tiempo medio igual a $T_r$, independiente de todos
    los anteriores.

    \par Se dice que el sistema falla cuando se tiene menos de $N$ máquinas
    funcionando en un momento dado o, equivalentemente, cuando posee más de $S$
    máquinas defectuosas en el taller de reparación.

    \par Para resolver el problema en cuestión, se desarrolló un algoritmo que
    simula la relación que se establece entre los lavarropas que se rompen,
    aquellos que se están arreglando y los lavarropas de repuesto hasta que el
    sistema falla.
    En esta simulación se considera un \textbf{evento} cuando se rompe una máquina
    o cuando se termina de arreglar una.
    Entonces, si se rompe una máquina, y no puede ser atendida por el
    técnico porque éste está arreglando otra, ésta se considera en la cola de
    espera para ser atendida. Una vez que un operario termina de arreglar un
    máquina, o bien comienza a reparar otra o bien se considera libre.
