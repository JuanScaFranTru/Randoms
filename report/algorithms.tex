\section{Algoritmo y descripción de las variables}

    \par En este trabajo se desarrolló un algoritmo que
    generaliza la idea de el problema propuesto. El algoritmo retorna el tiempo
    que pasa hasta que el sistema deja de ser operativo, dada una cantidad
    arbitraria de técnicos reparadores, máquinas de repuesto y máquinas que deben
    estar funcionando en todo momento.

    \subsection{Parámetros del Algoritmo}

        \par Los parámetros del algoritmo son:
        \begin{itemize}
            \item{$N$: Cantidad de máquinas que deben estar funcionando en todo
              momento.}
            \item{$S$: Cantidad de máquinas de repuesto.}
            \item{$T_f$: Tiempo medio de falla de una máquina.}
            \item{$T_r$: Tiempo medio de reparación de una máquina por un
              operador.}
            \item{$O$: Cantidad de operadores que reparan las máquinas.}
        \end{itemize}

        \par Tanto el tiempo de falla de una máquina como el tiempo que tarda un
        operario en arreglar una máquina, tienen distribución exponencial con
        media $T_f$ y $T_r$ respectivamente.


    \subsection{Variables internas del Algoritmo}

        Por otro lado, se utilizaron las siguientes variables:
        \begin{itemize}
            \item{$t$: Tiempo actual.}
            \item{$failure\_times$: Lista de los tiempos de fallos de máquinas.}
            \item{$fix\_times$: Lista de los tiempos en los que se finalizan los
              arreglos de las máquinas.}
            \item{$fixing$: Cantidad de operarios que están arreglando
              máquinas o equivalentemente, la cantidad de máquinas que están
              siendo arregladas.}
            \item{$broken$: Cantidad de máquinas que están rotas (incluye
              aquellas que están siendo arregladas).}
        \end{itemize}

    \begin{algorithm} [H]
    \caption{Simulation($N$, $S$, $O$, $T_f$, $T_r$)}
    \label{alg}
    \begin{algorithmic} [1]
    \STATE $ t \coloneqq  0 $
    \STATE $ broken \coloneqq  0 $
    \STATE $ fixing \coloneqq  0 $
    \STATE $ failure\_times \coloneqq \lbrack  \mathcal{E}(1/T_f)_0, \dots , \mathcal{E}(1/T_f)_{N-1} \rbrack $
    \STATE $ fix\_times \coloneqq  \lbrack \infty_0, \dots, \infty_{O-1}\rbrack $

    \WHILE{$ broken \leq S $}
    \IF{$ failure\_times_0 < fix\_times_0 $}

      \STATE $ t \coloneqq failure\_times_0$
      \STATE $ broken \coloneqq broken + 1 $

      \IF{$ broken \leq S $}
        \STATE $ failure\_times \coloneqq \lbrack X_1, \dots, X_{N-1}, t +
         \mathcal{E}(1/T_f)\rbrack $
        \STATE Sort $ failure\_times $
      \ENDIF

      \IF{$ broken > fixing $ and $ fix\_times_{N-1} = \infty $}
        \STATE $ fix\_times \coloneqq \lbrack Y_0, \dots, Y_{O-2}, t + \mathcal{E}(1/T_r)\rbrack $
        \STATE Sort $ fix\_times $
      \ENDIF

    \ELSE

      \STATE $ t \coloneqq fix\_times_0 $
      \STATE $ broken \coloneqq broken - 1 $
      \STATE $ fixing \coloneqq fixing - 1 $

      \IF{$ broken = fixing $}
        \STATE $ fix\_times \coloneqq \lbrack Y_1, \dots, Y_{O-1}, \infty \rbrack $
      \ENDIF

      \IF{$ broken > fixing $}
        \STATE $ fix\_times \coloneqq \lbrack Y_1, \dots, Y_{O-1}, t + \mathcal{E}(1/T_r)\rbrack $
        \STATE $ fixing \coloneqq fixing + 1 $

      \ENDIF
      \STATE Sort $ fix\_times $

    \ENDIF
    \ENDWHILE
    \RETURN $ t $
    \end{algorithmic}
    \end{algorithm}

    \par El algoritmo mantiene dos listas ($ failure\_times$ y $fix\_times $)
    con los tiempos de los eventos ordenados de menor a mayor. Dado que están
    ordenadas, $ min(failure\_times_0, fix\_times_0) $ representa el próximo
    evento que ocurrirá. Los tiempos de falla se inicializan con valores
    generados a partir de una variable aleatoria exponencial con media $T_f$.
    Los tiempos de reparación se inicializan todos en infinito dado que no
    existen lavarropas en reparación.

    \par La simulación del lavadero se realiza hasta que éste deje de funcionar,
    es decir hasta que la cantidad de máquinas descompuestas sea mayor a la
    cantidad de máquinas de repuesto i.e., $ broken > S $.

    \par Si se descompone un lavarropas, se lo reemplaza por uno de repuesto y
    se calcula para este último el momento en el que fallará. Además, si un
    operario está libre comenzará a arreglar el lavarropa; por lo que también se
    calcula el tiempo que tardará.

    \par Si un operario termina de arreglar un lavarropas y hay un lavarropas
    descompuesto, se pone a trabajar inmediatamente y se calcula el tiempo que
    tardará en completar su tarea. Si no hay un lavarropas descompuesto, se
    marca al operario como libre. Para indicar que está libre se actualiza su
    tiempo a infinito. Esto significa que el próximo evento relacionado con la
    finalización de una reparación realizada por este operario nunca ocurrirá.

\pagebreak
